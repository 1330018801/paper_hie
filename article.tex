%%
%% Copyright 2007, 2008, 2009 Elsevier Ltd
%%
%% This file is part of the 'Elsarticle Bundle'.
%% ---------------------------------------------
%%
%% It may be distributed under the conditions of the LaTeX Project Public
%% License, either version 1.2 of this license or (at your option) any
%% later version.  The latest version of this license is in
%%    http://www.latex-project.org/lppl.txt
%% and version 1.2 or later is part of all distributions of LaTeX
%% version 1999/12/01 or later.
%%
%% The list of all files belonging to the 'Elsarticle Bundle' is
%% given in the file `manifest.txt'.
%%

%% Template article for Elsevier's document class `elsarticle'
%% with numbered style bibliographic references
%% SP 2008/03/01
%%
%%
%%
%% $Id: elsarticle-template-num.tex 4 2009-10-24 08:22:58Z rishi $
%%
%%
\documentclass[preprint,11pt]{elsarticle}

%% Use the option review to obtain double line spacing
%% \documentclass[preprint,review,12pt]{elsarticle}

%% Use the options 1p,twocolumn; 3p; 3p,twocolumn; 5p; or 5p,twocolumn
%% for a journal layout:
%% \documentclass[final,1p,times]{elsarticle}
%% \documentclass[final,1p,times,twocolumn]{elsarticle}
%% \documentclass[final,3p,times]{elsarticle}
%% \documentclass[final,3p,times,twocolumn]{elsarticle}
%% \documentclass[final,5p,times]{elsarticle}
%% \documentclass[final,5p,times,twocolumn]{elsarticle}

%% if you use PostScript figures in your article
%% use the graphics package for simple commands
%% \usepackage{graphics}
%% or use the graphicx package for more complicated commands
%% \usepackage{graphicx}
%% or use the epsfig package if you prefer to use the old commands
%% \usepackage{epsfig}

%% The amssymb package provides various useful mathematical symbols
\usepackage{amssymb}
%% The amsthm package provides extended theorem environments
%% \usepackage{amsthm}

%% The lineno packages adds line numbers. Start line numbering with
%% \begin{linenumbers}, end it with \end{linenumbers}. Or switch it on
%% for the whole article with \linenumbers after \end{frontmatter}.
%% \usepackage{lineno}

%% natbib.sty is loaded by default. However, natbib options can be
%% provided with \biboptions{...} command. Following options are
%% valid:

%%   round  -  round parentheses are used (default)
%%   square -  square brackets are used   [option]
%%   curly  -  curly braces are used      {option}
%%   angle  -  angle brackets are used    <option>
%%   semicolon  -  multiple citations separated by semi-colon
%%   colon  - same as semicolon, an earlier confusion
%%   comma  -  separated by comma
%%   numbers-  selects numerical citations
%%   super  -  numerical citations as superscripts
%%   sort   -  sorts multiple citations according to order in ref. list
%%   sort&compress   -  like sort, but also compresses numerical citations
%%   compress - compresses without sorting
%%
%% \biboptions{comma,round}

% \biboptions{}


%% \journal{Nuclear Physics B}

\begin{document}

\begin{frontmatter}

%% Title, authors and addresses

%% use the tnoteref command within \title for footnotes;
%% use the tnotetext command for the associated footnote;
%% use the fnref command within \author or \address for footnotes;
%% use the fntext command for the associated footnote;
%% use the corref command within \author for corresponding author footnotes;
%% use the cortext command for the associated footnote;
%% use the ead command for the email address,
%% and the form \ead[url] for the home page:
%%
%% \title{Title\tnoteref{label1}}
%% \tnotetext[label1]{}
%% \author{Name\corref{cor1}\fnref{label2}}
%% \ead{email address}
%% \ead[url]{home page}
%% \fntext[label2]{}
%% \cortext[cor1]{}
%% \address{Address\fnref{label3}}
%% \fntext[label3]{}

\title{Health Information Exchange Practice in Typical Chinese County-level Region}

%% use optional labels to link authors explicitly to addresses:
%% \author[label1,label2]{<author name>}
%% \address[label1]{<address>}
%% \address[label2]{<address>}

\author{Wang Lijun\corref{cor1}}
\ead{wanglijun@cernet.edu.cn}

\address{Institute for Network Science and Cyberspace, Tsinghua University, Beijing, China}
\cortext[cor1]{Corresponding author}


\begin{abstract}
%% Text of abstract
In this paper, we introduce our practice of HIE project.
\end{abstract}

\begin{keyword}
%% keywords here, in the form: keyword \sep keyword

%% MSC codes here, in the form: \MSC code \sep code
%% or \MSC[2008] code \sep code (2000 is the default)
Health Information Exchange \sep Cloud Computing \sep Big Data
\end{keyword}

\end{frontmatter}

%%
%% Start line numbering here if you want
%%
% \linenumbers

%% main text
\section{Introduction}
Electronic health information(HIE) allows doctors, nurses, pharmacists, other health care providers and patients to appropriately access and securely share a patient's vital medical information electroniclly - improving the speed, quality, safety and cost of patient care.

While electronic health inforamtion exchange cannot replace provider-patient communication, it can greatly improve the completeness of patient's records, as past history, current medications and other informatin is jointly reviewed during visits.

The US government as well as other national governments are making substantial investments to further the growth of HIE. Many local governments and individual organization are also following.

Accelerating the availability of electronic health information to guide dicision making and promoting care coordiation are core strategies of HHS efforts to imporve US health care delivery.

More than three-quarters of non-federal acute care hospitals electronically exchanged laboratory results, radiology reports, clinical care summaries, and medication list with any outside providers.

In the past, competition among hospital and cross-vendor interoperability were cited as barriers to exchange among outside hospitals; however, we found that this gap has narrowed.

In this paper, we are to share the experience gained in HIE practice and to point out some practical and potencial utilities of HIE to imporve regional health care.

In such a typical county-level region, multi-level hospitals form a medical service market with almost totoally competition. So the care provider are willing to attempt to new things, including HIE. The hospital would like to take advangate over other competitors.

The local government plays the most important role in the HIE practice. Though in a competitive market, the health care providers have resort to the local health administration, such as the medical reimbursement and public health service payment. 

\section{System architecture}

\section{Data structure}

\section{Applicatoin}

\section{Lession from the practice}
\subsection{subsection-1}
Lesson learned from the practice.
Lesson learned from the practice.
Lesson learned from the practice.
Lesson learned from the practice.
\subsection{subsection-2}
Lesson learned from the practice.
Lesson learned from the practice.
Lesson learned from the practice.
Lesson learned from the practice.
\section{Discussion}

\section{Conclusion}

\section{Knowledgements}
\label{aaa}

\section{Reference}

%% The Appendices part is started with the command \appendix;
%% appendix sections are then done as normal sections
%% \appendix

%% \section{}
%% \label{}

%% References
%%
%% Following citation commands can be used in the body text:
%% Usage of \cite is as follows:
%%   \cite{key}         ==>>  [#]
%%   \cite[chap. 2]{key} ==>> [#, chap. 2]
%%

%% References with bibTeX database:

\bibliographystyle{elsarticle-num}
\bibliography{<your-bib-database>}

%% Authors are advised to submit their bibtex database files. They are
%% requested to list a bibtex style file in the manuscript if they do
%% not want to use elsarticle-num.bst.

%% References without bibTeX database:

% \begin{thebibliography}{00}

%% \bibitem must have the following form:
%%   \bibitem{key}...
%%

% \bibitem{}

% \end{thebibliography}


\end{document}

%%
%% End of file `elsarticle-template-num.tex'.